\documentclass[12pt,a4paper]{moderncv}
\moderncvtheme[blue]{classic}
\usepackage[utf8]{inputenc}
\usepackage{pifont}
\usepackage[inline]{enumitem}
\usepackage{tikz}
\usetikzlibrary{tikzmark}
% Marge aux 4 coins de la page, ici elles sont réduites pour gagner de la place
\usepackage[top=0.5cm, bottom=0.5cm, left=1.0cm, right=1.0cm]{geometry}
% Largeur de la colonne de gauche pour les dates
\setlength{\hintscolumnwidth}{3.0cm}
\firstname{Michaël}
\familyname{Rollin}
\title{Ingénieur Système}
\address{\tikzmark{start} 37 rue de France}{06000 Nice}
\email{michael.rollin@orange.fr}
\homepage{https://github.com/mickahell}
\mobile{06 76 06 02 43}
\extrainfo{27 ans\tikzmark{end}}

\begin{document}
\begin{tikzpicture}[remember picture,overlay]
\fill[color1!35]
  (current page.north west) rectangle ([yshift=-1em]current page.east|-{pic cs:end});
\end{tikzpicture}
\maketitle
% Marge négative entre le titre et la partie expérience, pour gagner de la place
\vspace*{-3\baselineskip}

\section{Expériences}
\cventry{Mars 2020 \\à Aujourd'hui}{Chargé d'étude en quantum computing}{Altran}{}{}{
\begin{itemize}[topsep=1pt, itemsep=0pt, partopsep=0pt, parsep=0pt]
\item Chargé de création de documents de formation et de tuto en quantum computing à destination de personne ayant un profil de développeur ou d'ingénieur en informatique.\end{itemize}}
\cventry{Juillet 2018\\à Aujourd'hui}{Administrateur système}{Orange par Altran}{Sophia-Antipolis}{France}{
Ops en charge des serveurs de production sur les projets Meta \& Search. Projets consistant à la récupération et au traitement des métadonnées en lien avec la TV d’Orange et au moteur de recherche.
\begin{itemize}[topsep=2pt, itemsep=1pt, partopsep=1pt, parsep=1pt]
\item Dockeurisation du système de traitement des métadonnées
\item Migration d'une architecture physique en une architecture serverless
\item Création d’une nouvelle architecture en microservices
\item Automatisation de processus via Ansible
\end{itemize}}
\cventry{Juin 2017 \\à Juin 2019}{Web Project Administrator}{Erasmus Student Network France}{}{}{
Responsable de la stratégie numérique du réseau ESN France ainsi que des outils et des projets numériques.
\begin{itemize}[topsep=2pt, itemsep=1pt, partopsep=1pt, parsep=1pt]
\item Responsable du projet européen \href{https://buddysystem.eu/fr/the-project}{Buddy System}
\item Création de différentes plateformes : intranet, webshop, ...
\end{itemize}}
\cventry{Mars 2017 \\à Septembre 2017}{Ingénieur Cybersécurité}{Ekium}{Lyon}{}{
\begin{itemize}[topsep=1pt, itemsep=0pt, partopsep=0pt, parsep=0pt]
\item Création d'un architecture de cybersécurité incluant un système de supervision pour une usine de dessalement d'eau de mer à Oman.\end{itemize}}

\section{Formations}
\cventry{2016 -- 2017}{Master 2}{ESAIP Dijon (21)}{Chef de projet international en systèmes numériques}{}{}
\cventry{2016}{Master 1}{LaSalle Cuernavaca (Mexique)}{Cibernética}{}{}
\cventry{2015}{Licence 3}{Politechnika Varsovie (Pologne)}{Electrical Engineering}{}{}
\cventry{2014}{BTS}{St Joseph Dijon}{Systèmes électroniques}{}{}

\section{Certifications}
\cventry{Juillet 2021}{\href{https://www.credly.com/badges/7b27192c-7f23-4725-bd08-a27e7a76c7cb}{IBM Certified Associate Developer - Quantum Computation using Qiskit v0.2X}}{IBM}{}{}{}
\cventry{Juillet 2021}{\href{https://qiskit.org/events/summer-school/}{Qiskit Global Summer School}}{IBM}{}{}{2 semaines (41h) de cours et de TP en quantum machine learning.}
\cventry{Nov 20/Mai 21}{\href{https://www.credly.com/badges/a020adc5-df85-48ae-9cbf-2e8bed405a5f}{IBM Quantum Challenge - Advanced}}{IBM}{Badge de niveau avancé}{}{}

\section{Compétences en informatique}
\cvitem{\underline{OS}}{Ubuntu, CentOS, Windows Serveur 2008 \texttt{/} 2012 R2}
\cvitem{\underline{Programmation}}{Python, C++, QISkit, Pennylane}
\cvitem{\underline{Admin. services}}{Docker, Ansible, CI/CD, Vmware}
\cvitem{\underline{Electronique}}{MPLAB, PCAD, MATLAB}
\cvdoubleitem{\underline{Anglais}}{\ding{72}\ding{72}\ding{72}\ding{72}\ding{73} -- Toeic 795 \texttt{/} 990}{\underline{Espagnol}}{\ding{72}\ding{72}\ding{72}\ding{72}\ding{73}}{}

\section{Centres d'intérêt}
\cvitem{}{Création d'articles pour \href{https://fullstackquantumcomputation.tech}{\textit{Full-Stack Quantum Computation}}, \href{https://pintofscience.fr/equipe/Nice}{Event manager à Pint of Science}, 16 ans de Judo ceinture Noire 1er dan.}
\end{document}