\documentclass[12pt,a4paper]{moderncv}
\moderncvtheme[blue]{classic}
\usepackage[utf8]{inputenc}
\usepackage{pifont}
\usepackage[inline]{enumitem}
\usepackage{tikz}
\usetikzlibrary{tikzmark}
% Marge aux 4 coins de la page, ici elles sont réduites pour gagner de la place
\usepackage[top=0.5cm, bottom=0.5cm, left=1.0cm, right=1.0cm]{geometry}
% Largeur de la colonne de gauche pour les dates
\setlength{\hintscolumnwidth}{3.0cm}
\firstname{Michaël}
\familyname{Rollin}
\title{Ingénieur Système}
\address{\tikzmark{start}}{69000 Lyon}
\email{michael.rollin@orange.fr}
\homepage{github.com/mickahell}
%\mobile{+33 6 76 06 02 43}
\extrainfo{30 ans\tikzmark{end}}

\begin{document}
\begin{tikzpicture}[remember picture,overlay]
\fill[color1!35]
  (current page.north west) rectangle ([yshift=-1em]current page.east|-{pic cs:end});
\end{tikzpicture}
\maketitle
% Marge négative entre le titre et la partie expérience, pour gagner de la place
\vspace*{-3\baselineskip}

\section{Expériences}
\cventry{Mars 2020 \\à Aujourd'hui}{Qiskit Advocate}{Qiskit community}{}{}{
Représentant de la communauté Qiskit, contributeur opensource pour les projets Qiskit.
\begin{itemize}[topsep=1pt, itemsep=0pt, partopsep=0pt, parsep=0pt]
\item Developpement de projets pour la communauté Qiskit
\item Création d'images Docker pour l'intégration quantique et d'environnement de lab
\item Création d'automatisation pour la résolution de jeux.
\end{itemize}}
\cventry{Juillet 2022\\à Novembre 2022}{Ingénieur DevOps}{EDF \& Enedis by Shape-IT}{Lyon}{France}{
Ingénieur responsable du maintien d'une infrastructure de serveurs physique et virtuel. Et responsable de l'environnement systeme de la plateforme d'intervention Linky.
\begin{itemize}[topsep=2pt, itemsep=1pt, partopsep=1pt, parsep=1pt]
\item Migration d'application sur une nouvelle plateforme via Ansible
\item Maintien et developpement de l'architecture via Terraform
\item Developpement de nouveaux outils transverse
\item Automatisation de processus via GitLab CI/CD, Jenkins et les outils Argo
\end{itemize}}
\cventry{Juillet 2018\\à Juillet 2022}{Administrateur système}{Orange par Altran}{Sophia-Antipolis}{France}{
Ops en charge des serveurs de production sur les projets Meta \& Search. Projets consistant à la récupération et au traitement des métadonnées en lien avec la TV d’Orange et au moteur de recherche.
\begin{itemize}[topsep=2pt, itemsep=1pt, partopsep=1pt, parsep=1pt]
\item Dockeurisation du système de traitement des métadonnées
\item Migration d'une architecture physique en une architecture serverless
\item Création d’une nouvelle architecture en microservices
\item Automatisation de processus via Ansible
\end{itemize}}

\section{Formations}
\cventry{2016 -- 2017}{Master 2}{ESAIP Dijon (21)}{Chef de projet international en systèmes numériques}{}{}
\cventry{2016}{Master 1}{LaSalle Cuernavaca (Mexique)}{Cibernética}{}{}
\cventry{2015}{Licence 3}{Politechnika Varsovie (Pologne)}{Electrical Engineering}{}{}
\cventry{2014}{BTS}{St Joseph Dijon}{Systèmes électroniques}{}{}

\section{Certifications}
\cventry{Juillet 2021}{\href{https://www.credly.com/badges/7b27192c-7f23-4725-bd08-a27e7a76c7cb}{IBM Certified Associate Developer - Quantum Computation using Qiskit v0.2X}}{IBM}{}{}{}
\cventry{July 2021}{\href{https://qiskit.org/events/summer-school/}{Qiskit Global Summer School}}{IBM}{}{}{}
\cventry{2020/21/22}{\href{https://www.credly.com/badges/a020adc5-df85-48ae-9cbf-2e8bed405a5f}{IBM Quantum Challenge - Advanced}}{IBM}{Badge de niveau avancé}{}{}

\section{Compétences en informatique}
\cvitem{\underline{OS}}{Ubuntu, Rhel}
\cvitem{\underline{Programmation}}{Go, Python, C++, QISkit, Arduino}
\cvitem{\underline{Admin. services}}{Docker, Ansible, CI/CD, Terraform}
% \cvitem{\underline{Electronique}}{Arduino, MPLAB, PCAD, MATLAB}
\cvdoubleitem{\underline{Anglais}}{\ding{72}\ding{72}\ding{72}\ding{72}\ding{73} -- Toeic 795 \texttt{/} 990}{\underline{Espagnol}}{\ding{72}\ding{72}\ding{72}\ding{72}\ding{73}}{}

\section{Centres d'intérêt}
\cvitem{}{Création d'articles pour \href{https://fullstackquantumcomputation.tech}{\textit{Full-Stack Quantum Computation}}, \href{https://pintofscience.fr/equipe/Nice}{Event manager à Pint of Science}, 16 ans de Judo ceinture Noire 1er dan.}
\end{document}