\documentclass[12pt,a4paper]{moderncv}
\moderncvtheme[blue]{classic}
\usepackage[utf8]{inputenc}
\usepackage{pifont}
\usepackage[inline]{enumitem}
\usepackage{tikz}
\usetikzlibrary{tikzmark}
% Marge aux 4 coins de la page, ici elles sont réduites pour gagner de la place
\usepackage[top=0.5cm, bottom=0.5cm, left=1.0cm, right=1.0cm]{geometry}
% Largeur de la colonne de gauche pour les dates
\setlength{\hintscolumnwidth}{3.0cm}
\firstname{Michaël}
\familyname{Rollin}
\title{System Engineer}
\address{\tikzmark{start}}{69000 Lyon}
\email{michael.rollin@orange.fr}
\homepage{github.com/mickahell}
\mobile{+33 6 76 06 02 43}
\extrainfo{29 years\tikzmark{end}}

\begin{document}
\begin{tikzpicture}[remember picture,overlay]
\fill[color1!35]
  (current page.north west) rectangle ([yshift=-1em]current page.east|-{pic cs:end});
\end{tikzpicture}
\maketitle
% Marge négative entre le titre et la partie expérience, pour gagner de la place
\vspace*{-3\baselineskip}

\section{Experiences}
\cventry{March 2020 \\to Today}{Qiskit Advocate \& Studies manager in quantum computing}{Qiskit community}{}{}{
Advocate for the Qiskit community, opensource contributor for Qiskit projects and also Studies manager in QC for Altran/Capgemini to education and proof of concept purposes.
\begin{itemize}[topsep=1pt, itemsep=0pt, partopsep=0pt, parsep=0pt]
\item Developing the project for the Qiskit community
\item Create Docker images for quantum integration and lab environment
\item Creating automation program for solving game problems
\end{itemize}}
\cventry{July 2022 \\to Today}{DevOps Engineer}{EDF \& Enedis by Shape-IT}{Lyon}{France}{
Engineer responsible of maintaining an infrastructure on AWS servers. And responsible of the system environment of Linky intervention platform.
\begin{itemize}[topsep=2pt, itemsep=1pt, partopsep=1pt, parsep=1pt]
\item Migration application from a physical architecture to the AWS architecture Kubernetes
\item Maintaining AWS architecture using Terraform
\item Development of new tools to help internal development
\item Automatization using GitLab CI/CD, Jenkins and Argo tools
\end{itemize}}
\cventry{July 2018\\to July 2022}{System Engineer}{Orange by Altran}{Sophia-Antipolis}{France}{
Admin system responsible of the production servers for project Meta and the Search. Project about to getting and processing every metadata of the Orange TV and process those data for the websearch Orange.
\begin{itemize}[topsep=2pt, itemsep=1pt, partopsep=1pt, parsep=1pt]
\item Dockeurisation of the metadata processing system \& search system
\item Migration from a physical architecture to a virtual architecture in the cloud
\item Development of a new architecture in microservices
\item Processus automatisation with Ansible
\end{itemize}}

\section{Education}
\cventry{2016 -- 2017}{Master 2}{ESAIP Dijon (21)}{IT project management}{}{}
\cventry{2016}{Master 1}{LaSalle Cuernavaca (Mexico)}{Cibernética}{}{}
\cventry{2015}{Licence 3}{Politechnika Warsaw (Poland)}{Electrical Engineering}{}{}
\cventry{2014}{BTS}{St Joseph Dijon}{Electronics systems}{}{}

\section{Certifications}
\cventry{July 2021}{\href{https://www.credly.com/badges/7b27192c-7f23-4725-bd08-a27e7a76c7cb}{IBM Certified Associate Developer - Quantum Computation using Qiskit v0.2X}}{IBM}{}{}{}
\cventry{July 2021}{\href{https://qiskit.org/events/summer-school/}{Qiskit Global Summer School}}{IBM}{}{}{}
\cventry{2020/21/22}{\href{https://www.credly.com/badges/a020adc5-df85-48ae-9cbf-2e8bed405a5f}{IBM Quantum Challenge - Advanced}}{IBM}{Advanced badges}{}{}

\section{IT skills}
\cvitem{\underline{OS}}{Ubuntu, CentOS, Windows Server 2008 \texttt{/} 2012 R2}
\cvitem{\underline{Coding}}{Go, Python, C++, QISkit, Arduino}
\cvitem{\underline{Admin. services}}{Docker, Ansible, CI/CD, Vmware}
% \cvitem{\underline{Electronic}}{Arduino, MPLAB, PCAD, MATLAB}
\cvdoubleitem{\underline{English}}{\ding{72}\ding{72}\ding{72}\ding{72}\ding{73} -- Toeic 795 \texttt{/} 990}{\underline{Spanish}}{\ding{72}\ding{72}\ding{72}\ding{72}\ding{73}}{}

\section{Hobbies}
\cvitem{}{Writer for \href{https://fullstackquantumcomputation.tech}{\textit{Full-Stack Quantum Computation}}, \href{https://pintofscience.fr/equipe/Nice}{Event manager at Pint of Science}, 16 years of judo 1st dan black belt.}
\end{document}